%-------------------------------------------------
% FileName: prac-note.tex
% Author: Safin (zhaoqid@zsc.edu.cn)
% Version: 0.1
% Date: 2021-11-21
% Description: 实践记录
% Others: 
% History: origin
%------------------------------------------------- 

% 断页
\clearpage
\chapter{实践记录}

% 注意换行需要空一行
% 详细记录实践过程,以下文字仅仅用于演示公式,图片,源代码的写法

% 实践记录
\section{开发环境与依赖}\label{sec:env}
\begin{itemize}
  \item 操作系统:Linux(含 WSL/虚拟机环境均可),使用 \texttt{/dev/fb0} 帧缓冲设备。
  \item 编译与工具:\texttt{g++} 与 \texttt{CMake};编辑器 VSCode。
  \item 第三方库:\texttt{FFmpeg}(\texttt{libavformat}\,\texttt{libavcodec}\,\texttt{libavutil}\,\texttt{libswscale})用于解码 JPG/PNG/BMP/GIF;不支持时回退 PPM 解析。
\end{itemize}

\section{项目结构与模块}\label{sec:arch}
项目基于帧缓冲绘图,按“核心/图形”分层组织:
\begin{itemize}
  \item \texttt{src/core}:\texttt{CFramebuffer} 初始化帧缓冲、清屏、绘制像素与边框;\texttt{Point}/\texttt{Color} 提供几何与颜色基础类型。
  \item \texttt{src/graphics}:\texttt{Image} 通过 FFmpeg 统一解码到 RGB24;\texttt{fractals}(分形龙)、\texttt{shapes}(折线/圆/奥运五环)、\texttt{games}(贪吃蛇、冒泡排序可视化)。
  \item \texttt{src/main.cpp}:命令行入口,支持 \texttt{dragon}\,\texttt{olympic}\,\texttt{snake}\,\texttt{bubble}\,\texttt{image} 模式。
\end{itemize}

\section{帧缓冲绘图与基础实现}\label{sec:fb}
帧缓冲通过 \texttt{mmap} 映射显存,逐像素写入实现绘图:
\begin{itemize}
  \item 初始化:读取 \texttt{fb\_var\_screeninfo}/\texttt{fb\_fix\_screeninfo},计算屏幕尺寸与像素偏移。
  \item 绘制接口:\texttt{DrawPoint}/\texttt{DrawPoints} 以 \texttt{Point} 序列驱动渲染;\texttt{DrawBorder} 绘制边框辅助对齐。
  \item 图形抽象:\texttt{Geometry} 提供平移/旋转/缩放与点集生成;\texttt{Polyline}/\texttt{Circle} 等派生类实现对应算法(圆采用 Bresenham)。
\end{itemize}

\section{功能实现与展示}\label{sec:features}
项目实现并演示了图片显示、分形龙、奥运五环、贪吃蛇、冒泡排序五个功能。

\subsection{显示图片功能}
\textbf{思路}:使用 FFmpeg 将图片视作“视频流”,统一解码到 RGB24 缓冲,再转换为 \texttt{Color} 点阵写入帧缓冲;支持 GIF 动图按帧延时显示,PPM 作为兜底解析。

\textbf{运行}:\texttt{./program image /path/to/image.jpg}。

\begin{figure}[H]
  \centering
  \includegraphics[width=.85\textwidth]{\detokenize{[PC电脑]截图_2025-10-26_16-44-49_显示图片功能.png}}
  \caption{显示图片:FFmpeg 解码并渲染到帧缓冲}
\end{figure}

\subsection{分形龙功能}
\textbf{算法}:递归将线段按“垂直中点”劈分为左右两段,迭代步数决定细节。垂直中点通过叉积符号判定左右,保证龙曲线的自相似与连贯。关键接口:\texttt{GetVertexs()}、\texttt{GetVerticalCenter()}、\texttt{IsLeft()}。

\textbf{运行}:\texttt{./program dragon 10}(建议 1--15)。

\begin{figure}[H]
  \centering
  \includegraphics[width=.85\textwidth]{\detokenize{[PC电脑]截图_2025-10-26_16-43-05_分形龙功能.png}}
  \caption{分形龙:递归劈分生成点集,折线绘制}
\end{figure}

\subsection{奥运五环功能}
\textbf{实现}:五环的各环用 Bresenham 圆算法生成并渲染。WASD 平移、Q/E 旋转、-/= 缩放。

\textbf{运行}:\texttt{./program olympic},按键交互控制。

\begin{figure}[H]
  \centering
  \includegraphics[width=.82\textwidth]{\detokenize{[PC电脑]截图_2025-10-26_16-43-22_奥运五环功能.png}}
  \caption{奥运五环:圆形点集叠加,支持交互变换}
\end{figure}

\subsection{贪吃蛇功能}
\textbf{实现}:蛇体用 \texttt{deque} 存储,支持边界与自碰撞检测; WASD 控制、P 暂停、R 重置。

\textbf{运行}:\texttt{./program snake},按键交互控制。

\begin{figure}[H]
  \centering
  \includegraphics[width=.82\textwidth]{\detokenize{[PC电脑]截图_2025-10-26_16-43-47_贪吃蛇功能.png}}
  \caption{贪吃蛇:矩形段渲染与碰撞检测}
\end{figure}

\subsection{排序可视化(冒泡排序)}
\textbf{实现}:柱状图高度表示数值,实时高亮比较/交换元素,统计比较与交换次数;轮次结束与“未交换”优化提前停止。支持空格开始/暂停、R 重新生成、Q/ESC 退出。

\textbf{运行}:\texttt{./program bubble 30}(数组大小 1--100)。

\begin{figure}[H]
  \centering
  \includegraphics[width=.85\textwidth]{\detokenize{[PC电脑]截图_2025-10-26_16-44-14_排序功能.png}}
  \caption{冒泡排序可视化:比较/交换高亮与统计信息}
\end{figure}

\section{问题与解决}\label{sec:issues}
\begin{itemize}
  \item FFmpeg 依赖链接失败:按库顺序补齐 \texttt{-lavformat -lavcodec -lavutil -lswscale} 并安装开发包后解决。
  \item 帧缓冲权限:\texttt{/dev/fb0} 默认需 root,使用 \texttt{sudo} 或配置 udev 规则放宽访问。
  \item 交互输入:终端设为非阻塞(\texttt{termios}),渲染循环不中断;退出时恢复终端模式。
\end{itemize}

\section{运行与验证}\label{sec:run}
同一帧缓冲画布运行各功能,统一清屏与边框绘制;截图如上所示。命令行用法见 \texttt{src/main.cpp} 帮助文本,效果与预期一致。

 