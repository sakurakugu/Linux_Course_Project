%-------------------------------------------------
% FileName: prac-sum.tex
% Author: Safin (zhaoqid@zsc.edu.cn)
% Version: 0.1
% Date: 2021-11-21
% Description: 实践总结
% Others: 
% History: origin
%------------------------------------------------- 

% 断页
\clearpage 

\chapter{实践总结}

% 本章,简单总结实践完成之后的想法

\section{实践之前对Linux编程的理解}
在实践开始之前,我对 Linux 编程的理解主要停留在“系统调用 + 命令行工具”的层面:
编译/链接流程、文件/进程/信号的基本 API,及常见的 shell 使用。
对图形输出、设备文件(如 \texttt{/dev/fb0})以及多媒体库(FFmpeg)的工程化集成缺乏直观认识。
总体上更偏理论与书本示例,动手实现较少。


\section{实践之后对Linux编程的理解}
通过帧缓冲绘图与算法可视化的完整实践,对“Linux 一切皆文件”的理念有了更具体的感受:
\texttt{/dev/fb0} 作为字符设备可直接 \texttt{mmap} 到用户空间,逐像素写入即可完成图形渲染;
FFmpeg 作为成熟生态可无缝解码多种图片/动图格式,工程中应关注链接顺序与依赖;
同时,终端非阻塞输入与资源清理(退出恢复终端模式、释放解码上下文)体现出“可运行、可维护”的工程意识。

\section{对今后编程的想法}
实践让我更愿意以“可视化 + 可交互”的方式验证算法与数据结构:
把抽象的过程转化为图形与动画,能更快定位问题并提升表达清晰度。
后续在课程项目中,会优先搭建最小可运行版本(MVP),逐步扩展功能,并在关键模块旁加入可视化或统计输出,形成快速反馈闭环。

\section{对今后使用Linux系统的想法}
在桌面或服务器环境中更主动使用 Linux 开发链路:
熟练使用 \texttt{man} 与源码阅读定位问题,利用包管理与编译选项处理依赖冲突;
图形或多媒体相关任务优先考虑成熟库(如 FFmpeg),在理解接口前提下避免“重复造轮子”。
同时,会更早关注权限与设备访问(如 udev 规则),减少运行时的阻碍。


\section{其它想法} 
本次项目把“底层设备 + 算法可视化 + 文档排版”串联成一个完整闭环——
代码、效果与报告相互印证,帮助形成系统化的工程观。
后续希望在此基础上尝试更复杂的图形管线(如 GPU/OpenGL)或更多算法可视化(排序、路径规划、分形等)。
 
 


 