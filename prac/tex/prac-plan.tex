%-------------------------------------------------
% FileName: prac-plan.tex
% Author: Safin (zhaoqid@zsc.edu.cn)
% Version: 0.1
% Date: 2021-11-21
% Description: 实践计划
% Others: 
% History: origin
%------------------------------------------------- 

% 此实践计划一般由老师设置固定的内容
% 注意换行需要空一格

% 实践目的
\practicegoal{
    通过 Linux 帧缓冲编程与图形/算法实践,掌握底层绘图原理、
    终端非阻塞交互、以及 FFmpeg 图像解码在工程中的集成方法;
    在实现分形龙、奥运五环、贪吃蛇与冒泡排序可视化的过程中,
    强化数据结构与算法的动手能力,为后续系统与图形课程奠定基础。
}

% 实践安排
\practiceschedule{
    完成一个基于 framebuffer 的绘图系统,以模块化方式实现以下功能:
    \begin{enumerate}[label=(\arabic*)]
        \item 图片显示(含 GIF 多帧);
        \item 分形龙生成与绘制;
        \item 奥运五环布局与交互变换;
        \item 贪吃蛇游戏;
        \item 冒泡排序可视化与统计。
    \end{enumerate}
    最终提交可运行程序与实践报告。
}

% 实践内容
\practicework{
    \begin{enumerate}[label=第\arabic*天:]
        \vspace{1em}
        \item 完成 framebuffer 初始化与像素绘制,封装 \texttt{CFramebuffer};
        \item 实现 \texttt{Geometry}/\texttt{Polyline}/\texttt{Circle} 基类与 Bresenham 圆;
        \item 集成 FFmpeg,完成图片显示模块(含 GIF 多帧);
        \item 实现分形龙模块与参数化控制,编写实践记录;
        \item 实现奥运五环与交互(平移/旋转/缩放),完善文档;
        \item 完成贪吃蛇与碰撞检测;
        \item 实现冒泡排序可视化与统计,统一风格与边框;
        \item 整理实践报告,编译并检查排版与截图。
        \vspace{1em}
    \end{enumerate}
}

% 实践指导书
\practiceguide{
    \begin{enumerate}[label=\arabic*.]
        \item Linux 系统 \texttt{man} 手册;
        \item FFmpeg 官方文档;
        \item Linux 系统及编程基础,清华大学出版社;
        \item XeLaTeX 文档。
    \end{enumerate}
}



% 以下不用改动-------------------------------------
% 断页
\clearpage
% 页码从1开始计数
\setcounter{page}{1} 
% 阿拉伯数字显示页码
\pagenumbering{arabic}

\chapter{实践计划}

% 根据以上的参数 生成实践计划
\makepracplan
	
	

    
